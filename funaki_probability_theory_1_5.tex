\documentclass[12pt,a4paper]{jsarticle}
\usepackage{mymacro}

\begin{document}

\title{舟木確率論1.5ノート}
\author{katatoshi}
\maketitle

$p, q \in [0, 1]$, $p + q = 1$とし, 
\begin{gather*}
    \Omega = \{a, b\}^n = \{(\omega_1, \cdots, \omega_n) \mid \omega_i \in \{a, b\}\}, \\
    \sharp(a; \omega) = \sharp\{k \mid 0 \leq k \leq n, \omega_k = a\}, \\
    \sharp(b; \omega) = \sharp\{k \mid 0 \leq k \leq n, \omega_k = b\}.
\end{gather*}
とする.$A \subseteq \Omega$に対して
\begin{equation*}
    P(A) = \sum_{\omega \in A} p^{\sharp(a; \omega)} q^{\sharp(b; \omega)}
\end{equation*}
とする.

このとき,$P(\Omega) = 1$である.実際,
\begin{align*}
    P(\Omega)
    &= \sum_{\omega \in \Omega} p^{\sharp(a; \omega)} q^{\sharp(b; \omega)} \\
    &= \sum_{k = 0}^n \sharp\{\omega \in \Omega \mid \sharp(a; \omega) = k\} p^k q^{(n - k)}
\end{align*}
であり,$\sharp(a; \omega) = k$であるような$\omega$の数は$\omega_1, \cdots, \omega_n$の中から$k$個選ぶ選び方の数だから$\displaystyle \binom{n}{k}$である.したがって,二項定理より,
\begin{align*}
    P(\Omega)
    &= \sum_{k = 0}^n \binom{n}{k} p^k q^{n - k} \\
    &= (p + q)^n \\
    &= 1
\end{align*}
である.

他の例として$A_1 = \{\omega \in \Omega \mid \omega_1 = a\}$とすれば,
\begin{align*}
    P(A_1)
    &= \sum_{\omega \in A_1} p^{\sharp(a; \omega)} q^{\sharp(b; \omega)} \\
    &= \sum_{k = 0}^n \sharp\{\omega \in \Omega \mid \omega_1 = a, \sharp(a; \omega) = k\} p^k q^{n - k} \\
    &= \sum_{k = 1}^n \sharp\{\omega \in \Omega \mid \omega_1 = a, \sharp(a; \omega) = k\} p^k q^{n - k} \\
    &= \sum_{k = 0}^{n - 1} \sharp\{\omega \in \Omega \mid \omega_1 = a, \sharp(a; \omega) = k + 1\} p^{k + 1} q^{n - (k + 1)} \\
    &= p \sum_{k = 0}^{n - 1} \sharp\{\omega \in \Omega \mid \omega_1 = a, \sharp(a; \omega) = k + 1\} p^k q^{(n - 1) - k} \\
    &= p \sum_{k = 0}^{n - 1} \binom{n - 1}{k} p^k q^{(n - 1) - k} \\
    &= p(p + q)^{n - 1} \\
    &= p
\end{align*}
である.

\begin{thebibliography}{99}
\bibitem{舟木}
    舟木直久
    『確率論』
    朝倉書店, 2004.
\end{thebibliography}

\end{document}
