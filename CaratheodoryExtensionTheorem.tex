\documentclass[12pt,a4paper]{jsarticle}
\usepackage{mymacro}

\begin{document}

\title{Carath\'{e}doryの拡張定理}
\author{katatoshi}
\maketitle

集合$X$の部分集合族$\mathcal{G}$上で定義された関数$\mu: \mathcal{G} \to [0, \infty]$が\emph{前測度}(pre-measure)であるとは以下の性質をみたすことである.
\begin{enumerate}
    \item $\varnothing \in \mathcal{G}$ならば$\mu(\varnothing) = 0$
    \item $\mathcal{G}$の元の列$(G_i)_{i = 1}^\infty$が互いに素,すなわち$i \not= j$ならば$G_i \cap G_j = \varnothing$,であり$\bigcup_{i = 1}^\infty G_i \in \mathcal{G}$ならば$\mu\left(\bigcup_{i = 1}^\infty G_i\right) = \sum_{i = 1}^\infty \mu(G_i)$
\end{enumerate}
$\sigma$-加法族上の前測度は測度に他ならない.

集合$X$の部分集合族$\mathcal{S}$は以下の性質をみたすとき($X$上の)\emph{集合の半環}(semi-ring of sets)\footnote{半加法族と呼んでいるテキストもある.しかしsemi-ringをそのように呼んでしまうと次に述べるringを加法族と呼びたくなりalgebraと呼ばれる集合族と紛らわしい($\sigma$-algebraを$\sigma$-加法族と呼ぶのと同じようにalgebraを加法族と呼びたくなる).そこでこの文章ではsemi-ringを集合の半環と呼びringを集合の環と呼ぶことにした.なお,ring $\mathcal{R}$は$X \in \mathcal{R}$であるときalgebraと呼ばれる.ringは必ずしも$X$を含まないためringとalgebraは異なる概念である.}と呼ばれる.
\begin{enumerate}
    \item $\varnothing \in \mathcal{S}$
    \item $S, T \in \mathcal{S}$ならば$S \cap T \in \mathcal{S}$
    \item $S, T \in \mathcal{S}$ならば有限個の互いに素な$\mathcal{S}$の元$S_1, S_2, \cdots, S_n$が存在して$S \setminus T = \bigcup_{i = 1}^n S_i$
\end{enumerate}

集合$X$の部分集合族$\mathcal{R}$は以下の性質をみたすとき($X$上の)\emph{集合の環}(ring of sets)と呼ばれる.
\begin{enumerate}
    \item $\varnothing \in \mathcal{R}$
    \item $S, T \in \mathcal{R}$ならば$S \cup T \in \mathcal{R}$
    \item $S, T \in \mathcal{R}$ならば$S \setminus T \in \mathcal{R}$
\end{enumerate}

集合の環$\mathcal{R}$の元$S, T$について$S \cap T = S \setminus (S \setminus T) \in \mathcal{R}$が成り立つので,集合の環は共通部分についても閉じている.集合の環は集合の半環であり,$\sigma$-加法族は集合の環である.

$\mathfrak{R}$をすべての元が集合$X$上の集合の環であるような集合族とする.するとその共通部分$\bigcap \mathfrak{R} = \{S \mid \forall \mathcal{R} \in \mathfrak{R}(S \in \mathcal{R})\}$は再び集合$X$上の集合の環となる.

実際,任意の$\mathcal{R} \in \mathfrak{R}$について$\varnothing \in \mathcal{R}$であるから$\varnothing \in \bigcap \mathfrak{R}$となり(1)をみたす.次に,$S, T \in \bigcap \mathfrak{R}$ならば,任意の$\mathcal{R} \in \mathfrak{R}$について$S, T \in \mathcal{R}$であるから,任意の$\mathcal{R} \in \mathfrak{R}$について$S \cup T \in \mathcal{R}$である.よって,$S \cup T \in \bigcap \mathfrak{R}$となり(2)をみたす.(3)をみたすことは(2)と同様にして確認できる.

集合$X$の部分集合族$\mathcal{G}$に対して,$\mathcal{G}$を包むような$X$上の集合の環の全体の集合族を$\mathfrak{R}$とすると,$\bigcap \mathfrak{R}$は$\mathcal{G} \subseteq \bigcap \mathfrak{R}$をみたす集合$X$上の集合の環であり,任意の$\mathcal{R} \in \mathfrak{R}$について$\bigcap \mathfrak{R} \subseteq \mathcal{R}$である.すなわち,$\bigcap \mathfrak{R}$は$\mathcal{G}$を包むような$X$上の集合の環の中で最小のものであり,これを$\mathcal{G}$によって\emph{生成された}集合の環と呼び,ここでは$\rho(\mathcal{G})$で表すことにする.

\begin{proposition}
    $\mathcal{S}$を集合$X$上の集合の半環とすると
    \begin{equation*}
        \rho(\mathcal{S}) = \{S_1 \cup \cdots \cup S_n \mid n \in \bm{N}, S_1, \cdots, S_n \in \mathcal{S}\text{は互いに素}\}
    \end{equation*}
\end{proposition}
\begin{proof}
    右辺の集合を$\mathcal{U}$とおく.$\mathcal{S} \subseteq \mathcal{U} \subseteq \rho(\mathcal{S})$であるから,$\mathcal{U}$が集合の環であることを示せば,$\rho(\mathcal{S})$が$\mathcal{S}$を包む最小の集合の環であることから$\rho(\mathcal{S}) = \mathcal{U}$となる.

    $\mathcal{U}$が集合の環の性質(1), (2), (3)をみたすことを示す.$\varnothing \in \mathcal{S} \subseteq \mathcal{U}$であるから,$\mathcal{U}$は(1)をみたす.次に,$S = S_1 \cup \cdots \cup S_m$, $T = T_1 \cup \cdots \cup T_n \in \mathcal{U}$とおく.$\mathcal{U}$はその定義から互いに素な集合の和集合について閉じており,$S_i \cap T_j$, $i = 1, \cdots, m$, $j = 1, \cdots, n$は互いに素であるから
    \begin{equation*}
        S \cap T = (S_1 \cup \cdots \cup S_m) \cap (T_1 \cup \cdots \cup T_n) = \bigcup_{i = 1}^m \bigcup_{j = 1}^n (S_i \cap T_j) \in \mathcal{U}
    \end{equation*}
    となり,$\mathcal{U}$は共通部分について閉じている.$\mathcal{U}$が共通部分について閉じていることと,集合の半環の性質(3)より$S_i \setminus T_j \in \mathcal{U}$, $i = 1, \cdots m$, $j = 1, \cdots n$が成り立つことから
    \begin{equation*}
        S \setminus T = (S_1 \cup \cdots \cup S_m) \setminus (T_1 \cup \cdots \cup T_n) = \bigcup_{i = 1}^m \bigcap_{j = 1}^n (S_i \setminus T_j) \in \mathcal{U}
    \end{equation*}
    である.したがって,$\mathcal{U}$は(3)をみたす.$\mathcal{U}$は差集合,共通部分,互いに素な集合の和集合について閉じているので
    \begin{equation*}
        S \cup T = (S \setminus T) \cup (S \cap T) \cup (T \setminus S) \in \mathcal{U}
    \end{equation*}
    となり,$\mathcal{U}$は(2)をみたす.
    \qed
\end{proof}

\begin{proposition}
    $\mathcal{S}$を集合$X$上の集合の半環とし,$\mu:\mathcal{S} \to [0, \infty]$を$\mathcal{S}$上の前測度とする.このとき,$\mu$は集合の環$\rho(\mathcal{S})$上の前測度$\bar{\mu}:\rho(\mathcal{S}) \to [0, \infty]$に一意に拡張される.
\end{proposition}

集合$X$の冪集合$\mathcal{P}(X)$上で定義された関数$\mu^*: \mathcal{P}(X) \to [0, \infty]$が\emph{外測度}(outer measure)であるとは以下の性質をみたすことである.
\begin{enumerate}
    \item $\mu^*(\varnothing) = 0$
    \item (単調性)$A \subseteq B$ならば$\mu^*(A) \leq \mu^*(B)$
    \item ($\sigma$-劣加法性)$\mathcal{P}(X)$の元の列$(A_i)_{i = 1}^\infty$に対して,$\mu^*\left(\bigcup_{i = 1}^\infty A_i\right) \leq \sum_{i = 1}^\infty \mu^*(A_i)$
\end{enumerate}

\begin{proposition}\label{prop:tobemeasurable}
    $X$を集合とし,$\mu^*$を$\mathcal{P}(X)$上の外測度とする.$X$の部分集合族$\mathcal{A}^*$を
    \begin{equation*}
        \mathcal{A}^* = \{A \subseteq X \mid \forall Q \subseteq X(\mu^*(Q) = \mu^*(Q \cap A) + \mu^*(Q \setminus A))\}
    \end{equation*}
    と定義すると,$\mathcal{A}^*$は$\sigma$-加法族となる.また,$\mu^*$の$\mathcal{A}^*$への制限$\mu^*|_{\mathcal{A}^*}:\mathcal{A}^* \to [0, \infty]$は$(X, \mathcal{A}^*)$上の測度となる.
\end{proposition}

命題\ref{prop:tobemeasurable}の$\sigma$-加法族$\mathcal{A}^*$の元$A \in \mathcal{A}^*$を$\mu^*$-\emph{可測集合}という.

$\mathcal{G}$を集合$X$の部分集合族とする.$\mathcal{G}$の元の列$(G_i)_{i = 1}^\infty$が$X$の部分集合$A \in \mathcal{P}(X)$の$\mathcal{G}$-被覆であるとは,$A \subseteq \bigcup_{i = 1}^\infty G_i$が成り立つことをいう.$A \in \mathcal{P}(X)$の$\mathcal{G}$-被覆全体の集合を$\mathcal{C}(A)$とする.

\begin{proposition}\label{prop:inducedoutermeasure}
    $\mathcal{G}$を$\varnothing \in \mathcal{G}$であるような$X$の部分集合族とし,関数$\mu: \mathcal{G} \to [0, \infty]$を前測度とする\footnote{前測度でなくとも$\mu(\varnothing) = 0$でありさえすれば命題は成り立つが,前測度でない場合には関心がないため,$\mu$は前測度であると仮定する.}.このとき,関数$\mu^*:\mathcal{P}(X) \to [0, \infty]$を$A \in \mathcal{P}(X)$に対して
    \begin{equation*}
        \mu^*(A) = \inf \left\{\sum_{i = 1}^\infty \mu(S_i) \relmiddle| (S_i)_{i = 1}^\infty \in \mathcal{C}(A)\right\}
    \end{equation*}
    と定めると$\mu^*$は外測度となる.ただし$\inf \varnothing = \infty$とする.
\end{proposition}

\begin{proof}
    $G_i = \varnothing, i \in \bm{N}$とすると$(G_i)_{i = 1}^\infty \in \mathcal{C}(\varnothing)$であるから,$\mu^*(\varnothing) \leq \sum_{i = 1}^\infty \mu(G_i) = 0$.$\mu^*(\varnothing) \geq 0$であるから$\mu^*(\varnothing) = 0$である.

    $A \subseteq B$とすると$\mathcal{C}(B) \subseteq \mathcal{C}(A)$であるから
    \begin{align*}
        \mu^*(A)
        &= \inf \left\{\sum_{i = 1}^\infty \mu(G_i) \relmiddle| (G_i)_{i = 1}^\infty \in \mathcal{C}(A)\right\} \\
        &\leq \inf \left\{\sum_{i = 1}^\infty \mu(G_i) \relmiddle| (G_i)_{i = 1}^\infty \in \mathcal{C}(B)\right\} \\
        &= \mu^*(B)
    \end{align*}
    である.よって$\mu^*$は単調性をみたす.

    $A_i \in \mathcal{P}(X)$, $i \in \bm{N}$とする.$\mu^*(A_i) = \infty$となる$i \in \bm{N}$が存在するか,任意の$i \in \bm{N}$に対して$\mu^*(A_i) < \infty$であるが$\sum_{i = 1}^\infty \mu^*(A_i) = \infty$となる場合,$\mu^*(\bigcup_{i = 1}^\infty A_i) \leq \infty$より$\mu^*(\bigcup_{i = 1}^\infty A_i)$ $\leq$ $\sum_{i = 1}^\infty \mu^*(A_i)$となる.

    任意の$i \in \bm{N}$に対して$\mu^*(A_i) < \infty$であり,$\sum_{i = 1}^\infty \mu^*(A_i) < \infty$であるとする.このとき,任意の$i \in \bm{N}$に対して$\mathcal{C}(A_i) \not= \varnothing$が成り立つ.下限の性質から,任意の$\varepsilon > 0$に対して
    \begin{equation*}
        \sum_{j = 1}^\infty \mu(G_{ij}) \leq \mu^*(A_i) + \frac{\varepsilon}{2i}
    \end{equation*}
    となるような被覆$(G_{ij})_{j = 1}^\infty \in \mathcal{C}(A_i)$が各$i \in \bm{N}$に対して存在する.両辺の$i$についての和をとると
    \begin{equation*}
        \sum_{i = 1}^\infty \sum_{j = 1}^\infty \mu(G_{ij}) \leq \sum_{i = 1}^\infty \mu^*(A_i) + \varepsilon
    \end{equation*}
    となるので,$\sum_{i = 1}^\infty \sum_{j = 1}^\infty \mu(G_{ij}) < \infty$である.よって,杉浦\cite{杉浦}定理5.4より
    \begin{equation*}
        \sum_{i, j = 1}^\infty \mu(G_{ij}) = \sum_{i = 1}^\infty \sum_{j = 1}^\infty \mu(G_{ij}) < \infty
    \end{equation*}
    である.すなわち,正項二重級数$\sum_{i, j = 1}^\infty \mu(G_{ij})$は収束するので,杉浦\cite{杉浦}定理5.5より,$\bm{N}$から$\bm{N} \times \bm{N}$への全単射$\phi$を一つとると,一列化$\sum_{k = 1}^\infty \mu(G_{\phi(k)})$は収束し,
    \begin{equation*}
        \sum_{k = 1}^\infty \mu(G_{\phi(k)}) = \sum_{i, j = 1}^\infty \mu(G_{ij})
    \end{equation*}
    である.$\mathcal{G}$の元の列$(G_{\phi(k)})_{k = 1}^\infty$は$\bigcup_{i = 1}^\infty A_i$の$\mathcal{G}$-被覆であるから\footnote{$x \in \bigcup_{i = 1}^\infty A_i$ならば$x \in A_i$となる$i \in \bm{N}$が存在する.$(G_{ij})_{j = 1}^\infty$は$A_i$の被覆であるから,$x \in G_{ij}$となる$j \in \bm{N}$が存在する.したがって,$x$ $\in G_{\phi(\phi^{-1}(i, j))}$ $\subseteq \bigcup_{k = 1}^\infty G_{\phi(k)}$である.},$\mu^*$の定義より
    \begin{align*}
        \mu^*\left(\bigcup_{i = 1}^\infty A_i\right)
        &= \sum_{k = 1}^\infty \mu(G_{\phi(k)}) \\
        &\leq \sum_{i, j = 1}^\infty \mu(G_{ij}) \\
        &= \sum_{i = 1}^\infty \sum_{j = 1}^\infty \mu(G_{ij}) \\
        &\leq \sum_{i = 1}^\infty \mu^*(A_i) + \varepsilon
    \end{align*}
    である.$\varepsilon$は任意であったから,$\mu^*(\bigcup_{i = 1}^\infty A_i)$ $\leq$ $\sum_{i = 1}^\infty \mu^*(A_i)$となる.よって,$\mu^*$は$\sigma$-劣加法性をみたす.
    \qed
\end{proof}

命題\ref{prop:inducedoutermeasure}の外測度$\mu^*$を,前測度$\mu$から\emph{誘導された}外測度という.

\begin{proposition}\label{prop:extensionpowerset}
    $\mathcal{S}$を集合$X$上の集合の半環とし,$\mu$を$\mathcal{S}$上の前測度とする.このとき,$\mu$から誘導された外測度$\mu^*$は$\mu$の$\mathcal{P}(X)$への拡張となっている.
\end{proposition}

\begin{proposition}\label{prop:semiringismeasurable}
    $\mathcal{S}$を集合$X$上の集合の半環,$\mu$を$\mathcal{S}$上の前測度,$\mu^*$を$\mu$から誘導された外測度とする.このとき,$S \in \mathcal{S}$は$\mu^*$-可測集合である.
\end{proposition}

\begin{theorem}[Carath\'{e}odory]
    $\mathcal{S}$を集合$X$上の集合の半環とし,$\mu$を$\mathcal{S}$上の前測度とする.このとき,$\mu$は$\sigma(\mathcal{S})$上の測度へ拡張することができる.さらに,$\mathcal{S}$の元の単調増加列$(S_i)_{i = 1}^\infty$で,$S_i \uparrow X$かつ任意の$i \in \bm{N}$について$\mu(S_i) < \infty$をみたすようなものが存在するならば,$\sigma(\mathcal{S})$への拡張は一意である.
\end{theorem}

\begin{proof}
    (存在すること)$\mu$から誘導された外測度を$\mu^*$とし,$\mu^*$-可測集合全体の集合を$\mathcal{A}^*$とする.命題\ref{prop:semiringismeasurable}より$\mathcal{S} \subseteq \mathcal{A}^*$であり,命題\ref{prop:tobemeasurable}より$\mathcal{A}^*$は$\sigma$-加法族であるから,$\sigma(\mathcal{S}) \subseteq \sigma(\mathcal{A}^*) = \mathcal{A}^*$が成り立つ.再び命題\ref{prop:tobemeasurable}より,$\mu^*|_{\mathcal{A}^*}$は$\mathcal{A}^*$上の測度であるから,その$\sigma(\mathcal{S})$への制限$\mu^*|_{\sigma(\mathcal{S})}$は$\sigma(\mathcal{S})$上の測度である.定理\ref{prop:extensionpowerset}より,$\mu^*$は$\mu$の$\mathcal{P}(X)$への拡張になっているので,$\mu^*|_{\sigma(\mathcal{S})}$は$\mu$の$\sigma(\mathcal{S})$への拡張である.

    (一意であること)$\mathcal{S}$の元の単調増加列$(S_i)_{i = 1}^\infty$で,$S_i \uparrow X$かつ任意の$i \in \bm{N}$について$\mu(S_i) < \infty$をみたすようなものが存在するならば,集合の半環$\mathcal{S}$は共通部分について閉じているので,Schilling\cite{Schilling}の定理5.7(測度の一意性定理)より,$\mu$の$\sigma(\mathcal{S})$への拡張は一意である.
    \qed
\end{proof}

\begin{thebibliography}{99}
\bibitem{Dudley}
    R.M. Dudley,
    \textit{Real analysis and probability} 2nd ed.,
    Cambridge : Cambridge University Press , 2002.
\bibitem{Schilling}
    Ren\'{e} L. Schilling,
    \textit{Measures, integrals and martingales},
    Cambridge University Press, 2011.
\bibitem{岩田}
    岩田耕一郎
    『ルベーグ積分 : 理論と計算手法』
    森北出版, 2015.
\bibitem{杉浦}
    杉浦光夫
     『解析入門』
     東京大学出版会, 1980.
\end{thebibliography}

\end{document}
