\documentclass[12pt,a4paper]{jsarticle}
\usepackage{mymacro}

\newcommand{\nat}{\mathbf{N}}
\newcommand{\real}{\mathbf{R}}
\newcommand{\calF}{\mathcal{F}}

\begin{document}

\title{正項級数の和の分割について}
\author{katatoshi}
\maketitle

$P \subseteq \nat$に対して,$P$の有限部分集合全体の集合を$\calF(P)$で表わす.$P \subseteq \nat$, $P \not= \varnothing$から$\real$への写像$a\colon P \to \real$を数列と呼び,$(a_n)_{n \in P}$で表わす.$F \in \calF(P)$に対して$s_F = \sum_{n \in F} a_n$とするとき,$\calF(P)$から$\real$への写像$F \mapsto s_F$を級数と呼び,$\sum_{n \in P} a_n$で表わす.$s_F$を級数$\sum_{n \in P} a_n$の部分和と呼び,任意の$n \in \nat$に対して$a_n \geq 0$であるとき,級数$\sum_{n \in P} a_n$を正項級数と呼ぶ.以下,正項級数についてのみ考える.正項級数$\sum_{n \in P} a_n$に対して
\begin{equation*}
    \sup_{F \in \calF(P)} s_F
\end{equation*}
を正項級数$\sum_{n \in P} a_n$の和と呼び,同じく$\sum_{n \in P} a_n$で表わす.$\sup_{F \in \calF(P)} s_F < +\infty$であるとき正項級数$\sum_{n \in P} a_n$は収束するという.

\begin{proposition}\label{main_prop}
    $P, P_1, P_2 \subseteq \nat$, $P \not= \varnothing$, $P_1 \not= \varnothing$, $P_2 \not= \varnothing$, $P_1 \cap P_2 = \varnothing$, $P_1 \cup P_2 = P$とする.このとき正項級数$\sum_{n \in P} a_n$が収束するための必要十分条件は$\sum_{n \in P_1} a_n$と$\sum_{n \in P_2} a_n$が共に収束することである.そしてこの条件が成り立つとき
    \begin{equation*}
        \sum_{n \in P} a_n = \sum_{n \in P_1} a_n + \sum_{n \in P_2} a_n
    \end{equation*}
    である.
\end{proposition}
\begin{proof}
    (必要性)$F_1$を$\calF(P_1)$の任意の元,$F_2$を$\calF(P_2)$の任意の元とする.$F_1 \cup F_2 \in \calF(P)$より$s_{F_1} + s_{F_2} = s_{F_1 \cup F_2} \leq \sup_{F \in \calF(P)} s_F$であるから
    \begin{equation}
        \sum_{n \in P_1} a_n + \sum_{n \in P_2} a_n = \sup_{F_1 \in \calF(P_1)} s_{F_1} + \sup_{F_2 \in \calF(P_2)} s_{F_2} \leq \sup_{F \in \calF(P)} s_F = \sum_{n \in P} a_n \label{main_prop:ineq_suff}
    \end{equation}
    である.$\sum_{n \in P} a_n$は収束するので$\sum_{n \in P_1} a_n + \sum_{n \in P_2} a_n \leq \sum_{n \in P} a_n < +\infty$である.したがって$\sum_{n \in P_1} a_n$, $\sum_{n \in P_2} a_n$は収束する.

    (十分性)任意の$F \in \calF(P)$は$F = F_1 \cup F_2$, $F_1 \in \calF(P_1)$, $F_2 \in \calF(P_2)$と表わされるので$s_F = s_{F_1} + s_{F_2} \leq \sup_{F_1 \in \calF(P_1)} s_{F_1} + \sup_{F_2 \in \calF(P_2)} s_{F_2}$である.したがって
    \begin{equation}
        \sum_{n \in P} a_n = \sup_{F \in \calF(P)} s_F \leq \sup_{F_1 \in \calF(P_1)} s_{F_1} + \sup_{F_2 \in \calF(P_2)} s_{F_2} = \sum_{n \in P_1} a_n + \sum_{n \in P_2} a_n \label{main_prop:ineq_ness}
    \end{equation}
    である.$\sum_{n \in P_1} a_n$, $\sum_{n \in P_2} a_n$は収束するので$\sum_{n \in P} a_n \leq \sum_{n \in P_1} a_n + \sum_{n \in P_2} a_n < +\infty$である.したがって$\sum_{n \in P} a_n$は収束する.

    (\ref{main_prop:ineq_suff}),(\ref{main_prop:ineq_ness})より
    \begin{equation*}
        \sum_{n \in P} a_n = \sum_{n \in P_1} a_n + \sum_{n \in P_2} a_n
    \end{equation*}
    である.
    \qed
\end{proof}

\begin{corollary}
    $P \subseteq \nat$, $P_l \subseteq \nat$, $l = 1, \cdots, L, L \geq 2$, $P \not= \varnothing$, $P_l \not= \varnothing$, $l = 1, \cdots, L$, $P_l \cap P_{l'} = \varnothing$, $l \not= l'$, $P = \bigcup_{l = 1}^L P_l$とする.このとき正項級数$\sum_{n \in P} a_n$が収束するための必要十分条件は任意の$l$, $l = 1, \cdots, L$に対して$\sum_{n \in P_l} a_n$が収束することである.そしてこの条件が成り立つとき
    \begin{equation*}
        \sum_{n \in P} a_n = \sum_{l = 1}^L \sum_{n \in P_l} a_n
    \end{equation*}
    である.
\end{corollary}
\begin{proof}
    $L$についての数学的帰納法で証明する.$L = 2$ならば命題\ref{main_prop}より主張は正しい.

    $L > 2$とし,$L - 1$に対しては主張は正しいと仮定する.$Q = P_1 \cup \cdots \cup P_{L - 1}$とすると,$Q \not= \varnothing$, $Q \cap P_L = \varnothing$, $Q \cup P_L = P$である.いま,$\sum_{n \in P} a_n$が収束するならば,命題\ref{main_prop}より$\sum_{n \in Q} a_n$と$\sum_{n \in P_L} a_n$は共に収束する.したがって,帰納法の仮定より,任意の$l$, $l = 1, \cdots, L$に対して$\sum_{n \in P_l} a_n$は収束する.逆に,任意の$l$, $l = 1, \cdots, L$に対して$\sum_{n \in P_l} a_n$が収束するならば,帰納法の仮定より,$\sum_{n \in Q} a_n$は収束するので,命題\ref{main_prop}より$\sum_{n \in P} a_n$は収束する.

    命題\ref{main_prop}より$\sum_{n \in P} a_n = \sum_{n \in Q} a_n + \sum_{n \in P_L} a_n$であり,帰納法の仮定より$\sum_{n \in Q} a_n = \sum_{l = 1}^{L - 1} \sum_{n \in P_l} a_n$であるから
    \begin{equation*}
        \sum_{n \in P} a_n = \sum_{l = 1}^{L - 1} \sum_{n \in P_l} a_n + \sum_{n \in P_L} a_n = \sum_{l = 1}^L \sum_{n \in P_l} a_n
    \end{equation*}
    である.
    \qed
\end{proof}

\end{document}
